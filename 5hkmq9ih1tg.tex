\textbf{Detailed Results} \\
General evolutionary trends for stars with masses between the hydrogen fusion limit and 5 Msun. Break stars up into three mass categories that are largely defined by their structure and evolution throughout the pre-MS: 
\begin{itemize}
 \item {\it very low mass} -- remain fully convective throughout their pre-MS evolution,
 \item {\it low mass} -- develop a radiative zone during their pre-MS evolution, but end on the MS with a radiative core and convective outer envelope,
 \item {\it intermediate mass} -- develop a radiative zone during their pre-MS evolution, but end on the MS with a convective core and radiative envelope,
 \item {\it high mass} -- no discernible pre-MS evolution.
\end{itemize}
As with all definitions, there are some ambiguities, particularly for stars with masses in the range of 0.28 $M_$