Imagine compressing a gaseous sphere of radius $$by some infinitesimal amount \epsilonϵ​. As in the case of a collapsing gas cloud, whether the object remains stable against this small perturbation depends on the sound crossing time from the the surface to the core and its relation to the Kelvin-Helmholtz timescale. If the sound crossing timescale is shorter than the Kelvin-Helmholtz contraction time, then the sound wave generated at the surface by the small perturbation will effectively inform other regions of the star about the perturbation, at which point the gas pressure will increase in response to steepen the pressure gradient and stabilize the star against further gravitational collapse. Thus, the star momentarily reestablishes hydrostatic equilibrium. 
However, the continued existence of a temperature gradient with the star leads to a net outward flux of energy, which is eventually radiated into empty space. This constant loss of energy causes the gas to cool, decreasing the average velocity (and thus momentum) of the particles leading to a decrease in gas pressure. A decrease in pressure again causes the star to contract slightly, by some small fraction \epsilonϵ​, after which the star will again momentarily reestablish hydrostatic equilibrium. Based on this reasoning, young pre-MS stars are expected to slowly contract in quasi-hydrostatic equilibrium.
At each step during the quasi-hydrostatic collapse, the gas pressure gradient must continually increase to balance the gravitational force, which is constantly increasing as the star collapses. Given that the surface temperature is expected to remain roughly constant throughout this contraction phase, pressures and temperatures throughout the star's interior must continually increase, leading to an increasing degree of ionization throughout the interior. The quasi-hydrostatic contraction proceeds until the core temperature becomes hot enough to completely ionize the primary heavy element constituents, notably iron. Once iron is fully ionized, there is a significant drop in radiative opacity within the core. 
MASS ADDITION/REMOVAL ARGUMENTS
This suggests that a star's pre-MS contraction time should be inversely proportional to the initial mass at the end of the protostellar phase, i.e., higher mass stars should contract toward the MS more quickly than lower mass stars. 

EVIDENCE --- Image of young stellar association(s).
The question remains, do we have evidence to support this hypothesis? After all, it was developed using purely qualitative arguments based on basic physical principles. Figure XX shows a Hertzsprung-Russell Diagram for three young stellar populations: the Orion Nebula Cluster, the Upper Scorpius subgroup of the Scorpius-Centaurus OB Association, and the Pleiades open cluster. [maybe only one example?] Notice that the hotter and presumably more massive stars lie closer to theoretical predictions for the ZAMS than do the cooler stars, which tend to lie well above the ZAMS. However, the degree to which the cooler stars lie above the ZAMS is not constant and depends on which stellar population we consider, but hotter star remain near the ZAMS in each case, lending support to our intuitive reasoning.