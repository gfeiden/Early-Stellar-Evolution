
Stars undergoing quasi-hydrostatic contraction can be described by the canonical set of stellar structure equations at every point in time during their contraction. The stellar structure equations are the equation for mass continuity
\begin{equation}
    \frac{dm}{dr} = 4\pi r^2 \rho(r),
\end{equation}
the equation describing hydrostatic equilibrium
\begin{equation}
    \frac{dP}{dr} = -\frac{G m(r)\rho(r)}{r^2},
\end{equation}
Meanwhile, the temperature gradient is described by either by radiative diffusion
\begin{equation}
    \frac{dT}{dr} = -\frac{3}{16\pi ac}\frac{\kappa(r) \rho(r) L(r)}{T(r)^3 r^2}
\end{equation}
or the adiabatic temperature gradient characteristic of convection
\begin{equation}
    \frac{dT}{dr} = (1 - \gamma)\frac{T(r)}{P(r)}\frac{dP}{dr}.
\end{equation}
Energy must be conserved at every location within the star,
\begin{equation}
    \frac{dL}{dr} = 4\pi r^2 \rho(r) \epsilon(r).
\end{equation}