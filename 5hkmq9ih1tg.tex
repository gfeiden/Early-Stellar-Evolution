\textbf{Detailed Results} \\
General evolutionary trends for stars with masses between the hydrogen fusion limit and 5 Msun. Break stars up into three mass categories that are largely defined by their structure and evolution throughout the pre-MS: 
\begin{itemize}
 \item[] {\it very low mass} -- remain fully convective throughout their pre-MS evolution,
 \item[] {\it low mass} -- develop a radiative zone during their pre-MS evolution, but end on the main sequence with a radiative core and convective outer envelope
 \item[] {\it intermediate mass} -- develop a radiative zone during their pre-MS evolution, but end on the main sequence with a convective core and radiative envelope.
low-mass stars develop a radiative zone at some point during their evolution ending on the main sequence with a radiative core and convective envelope, and intermediate mass stars also develop a radiative interior, but arrive on the main sequence
\end{itemize}